\documentclass[twoside]{article}
\oddsidemargin -0.5 cm
\evensidemargin -0.5 cm
\textwidth 17 cm
\topmargin -1.5 cm
\textheight 24.5 cm
\usepackage{shortvrb}
\MakeShortVerb{\|}

\author{POUPIN Pierre, SIX Cyril}
\title{\textbf{Implementation of an allocator}}

\begin{document}
\maketitle
\section{Structure of the project - modified files}
\subsection{Overview}
Several files have been modified in this project :
\begin{itemize}
	\item The file |mem_alloc.c| : it contains our implementation of the
		allocator. Because we have several versions of this implementation,
		instead of delivering one file for each version, we preferred to use
		the preprocessing instruction |#ifdef MACRO|.\\
		This is more deeply described in the next subsection.
	\item The files |alloc.in| and |alloc.out| have also been modified because
		there were either some inconsistencies in them or maybe just
		differences in the conventions used (for example, in |alloc3.in|,
		requesting the program to allow 10 bytes, and expecting it to actually
		allow 20 bytes).\\
		As we do not know which conventions have been used for these tests, we
		chose to modify them.
	\item The |Makefile| also has been modified to include 2 rules to generate
		the two small programs we created to test the checking system for
		freeing, and the system telling if the user has freed all blocks in the
		end of the program.
\end{itemize}

\subsection{Macros}
Each macro refers to a version or particularity of the program. It is activated
with a |#define MACRO|.
\begin{itemize}
	\item |BEST_FIT|, |WORST_FIT| : activating one of them will replace the
		|first fit| strategy with the |best fit| or | worst fit|.
	\item |CHECK_END| : this implements the checking at the end of the program.
		It's deactivated by default, to ensure the tests alloc.in give the 
		expected output.
\end{itemize}

\subsection{Makefile}
Two more rules have been added :
\begin{itemize}
	\item |test_end_checking| : a small program that tests the ckecing at exit
	\item |test_free_checking| : a small program that tests the check whenever
		the user tries to free a block.
\end{itemize}

\section{The answers to the questions}
\subsection{Question III.1}
Knowing the list of free blocks, and having the size stored in each block,
we have precise knowledge of where are the busy block.\\

Indeed : if the block isn't stored in the free block list, then it's a busy
block. Also, if there are several busy blocks next to each other, with their
size, we can jump from one to the next one.\\

So to know the list of busy block, what we have to do is start from the
beginning of the memory, and jump to the next block using the size attribute :
\begin{itemize}
	\item if the block is in the free list, then it's a free block
	\item if it's not the case, then it's a busy block
\end{itemize}

In |mem_alloc.c|, we use this technique to know if the free request of a user
is correct.

\subsection{Question III.2}
As a user, we cannot use addresses that haven't been allocated first. Indeed,
the allocator has to store headers in each allocated block ; if the user were
able to write wherever he wants, it would then screw up the allocator, which
would lead to weird behaviours (seg fault, infinite loop..).\\

So it's more secured to forbid the user to do it.

\subsection{Question III.3}
Therefore, when a block is allocated, we should return the address that is
actually usable by the user. For example, if the user allocates 20 bytes, then
the allocator turn a 24 bytes block into a busy block, and it returns the
address of the 5th byte (assuming the header is 4 bytes long).

\subsection{Question III.4}
So the user always use the address of the usable content, not the address of
the header. As a consequence, the free function will always have the address
of usable content as argument ; the first step is then to jump back of 4
bytes for example if the header is 4 bytes.\\

We have to take care to keep consistency of the linked list ; so link the
previous free block to this new one for example.

\subsection{Question III.5}
If there exists a free block neighbour, instead of creating a new free
block, we can "expand" the neighbour ! So the problem occurs only when there's
no free neighbour, so when the 2 neighbours are busy block.\\

If we freed this block, as the header is too large (rest too small), it would
write to the adjacent block and corrupt the header of the neighbour. Which
would result in undefinite behaviour.\\

If we didn't free the block and did nothing, then this part of the memory would
become unusable in the future (also, it would break the assumption that every
block is either a free block or a busy block ; and we need this assumption to
make the "free checking" work).\\

That's why we chose the following : instead of freeing the block or not doing
anything at all, we merge the busy block to the left neighbour busy block.

\begin{itemize}
	\item The pro : it solves our problem !
	\item The con : the user can use this extra memory without knowing it. For 
		example he could declare an array of 20 elements, go to the 21 elements,
		and the program wouldn't crash, while it would do unexpected behaviour.
\end{itemize}

Here, as we don't have any checking occuring when trying to access memory,
this is not a problem. But it would be a problem in a more consequent OS 
project.\\

Recently, we solved the problem by forcing the allocation of a minimum size
so in any case there is enough place to write the free block header, so the
implementation of this particular case shouldn't be used anymore (but we left it
just in case).

\section{Other decisions and conventions adopted}
\subsection{Conventions}
\begin{description}
	\item[size :] We store the size of the entire block (so header + usable 
		size)
	\item[addresses :] We consider the address of the beginning of the block
		(header included). But, for the user interface, we return the address
		that is directly usable.
	\item[structures :] We left it as it was. So the free block structure has
		a size and a pointer to the next free block, and the busy block only
		has the size stored in its header.
\end{description}

\subsection{Freeing behaviour}
We distinguish 4 cases :
\begin{itemize}
	\item If there's no free neighbour : by default, if the size of the block
		isn't big enough to create a free block, we merge this one to the
		previous busy block. This behaviour can be changed though, so that this
		situation never happens, and we allocate a larger memory space if the
		user wants a too small size.
	\item There's either a left neighbour or a right neighbour : then we merge
		the freed space to this neighbour (no need to create a second free
		header).
	\item There are both a left neighbour and a right neighbour : then we merge
		both the newly freed space \emph{and} the right neighbour to the left
		neighbour.
\end{itemize}

\subsection{Allocating behaviour}
We allocate a block if there exists a free block such that the size is big
enough to contain a free block header + the requested size.\\

If it's not the case, then we allow enough space to contain the free block
header.
\end{document}
